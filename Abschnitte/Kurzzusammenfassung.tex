%Diese Arbeit handelt von einer Software-in-the-Loop Teststrategie für elektrische Antriebsfunktionen.
%TPT ist ein Programm, das ursprünglich bei der Daimler AG entwickelt wurde, um eingebettete Systeme zu testen.
Es werden zwei Themen als Software-in-the-Loop Teststrategie bearbeitet, nämlich ein Schnittstellentest und ein Äquivalenzklassentest.
Beide Themen werden mit TPT erarbeitet. TPT ist ein Programm der Firma PikeTec zum Testen von eingebetteten Systemen.
Die Arbeitspakete für den Schnittstellentest sind einen bestehenden Schnittstellentest zu analysieren, 
eine C Datei nachzustellen und Testfälle automatisiert in TPT generieren lassen.
% Ein bestehender Schnittstellentest wird analysiert. Eine C Datei wird für den Schnittstellentest in TPT 
% nachgestellt. Die Testfälle werden automtatisiert in TPT generiert.
Für den Äquivalenzklassentest wird ein Konzept erarbeitet, wie diese eingesetzt werden können.
Es geht zum Einen darum, Äquivalenzklassen zu nutzen, um bestehende Software-in-the-Loop Tests zu überprüfen.
Zum Anderen geht es um eine Selbstüberprüfung der Äquivalenzklassen, ob die definierten Äquivalenzklassen alle erreichbar sind.
Es wird auch aufgezeigt, dass durch das Definieren von Äquivalenzklassen eine gewisse Datenbank für Tests entsteht und 
diese auch für weitere Tests eingesetzt werden kann.

% dass dadurch für Signale systematisch Werte festgelegt werden und somit eine Datenbasis angelegt wird,
% die auch für weitere Testfälle eingesetzt werden können.

% um eine Datenbasis zu erreichen, indem Werte definiert sind, die später auch für andere Tests eingesetzt werden
% können.
