


\ac{zf} ist ein Technologiekonzern mit rund 153.500 Mitarbeitern (Stand 2020) weltweit in den Bereichen Mobilität
und Industrietechnik. ZF übernahm zuletzt WABCO und konnte dadurch ihre Kompetenz im Bereich von Technologien
für schwere Nutzfahrzeuge, Busse und Trailer steigern. ZF gibt jährlich sieben Prozent des Umsatzes (Stand 2020) in 
Forschung und Entwicklung aus. ZF will die Veränderungen in der Mobilitätsbranche 
% spürt Veränderungen in der Mobilitätsbranche und will die Zukunft 
mit der
Strategie \glqq Next Generation Mobility\grqq{} vorantreiben. Das Ziel der Strategie ist eine einfache, saubere
Mobilität, die automatisiert, komfortabel und bezahlbar ist. Sie soll für jedermann und überall erreichbar sein.
Für die Bachelorarbeit bin ich in der Abteilung Softwareentwicklung elektrische
Antriebsfunktionen tätig \cite[vgl.][]{zf}.

% Quelle: https://www.zf.com/mobile/de/company/company.html    (in Literaturrecherche gespeichert)
% Abruf 28.01.22 um 14:22 Uhr