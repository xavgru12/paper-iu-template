Ein Ziel der Arbeit ist einen bestehenden Schnittstellentest in TPT zu überführen und zu optimieren.
TPT soll als einheitliche Testumgebung eingeführt werden. Dabei werden bestehende Tests, so auch der Schnittstellentest,
in TPT überführt. Der Schnittstellentest hat Potenziale in der Laufzeit (siehe Kapitel 3), was opimiert werden soll.
Ein zweites Ziel ist ein Konzept für den Äquivalenzklassentest in TPT zu erarbeiten (siehe Kapitel 6). 


% Optimieren des Quicktests, Einführung in TPT\\
% Konzept  und Möglichkeiten(Testszenarien) für Äquivalenzklassentest

% KEINE ZIELE?????\\
% Ein Ziel ist die Umsetzung.\\
% Unterschiede von state of the art zu status quo erkennen.\\
% Umsetzung möglichst nahe an state of the art.\\
% Quicktest optimieren, Äquivalenzklassentest einführen.