%Quelle Dissertattion TPT\\
% Unter dem Titel \glqq Systematischer Test des kontinuierlichen Verhaltens von eingebetten Systemen\grqq{} wurde \ac{tpt} im Rahmen
% einer Dissertation von Eckard Lehmann bei der Daimler AG entwickelt. 
Eingebettete Systeme zeichnen sich aus, dass sie über kontinuierliche Größen,
etwa Sensoren, mit der Umgebung verbunden sind. \ac{tpt} wurde im Rahmen einer Dissertation von Eckard Lehmann bei der Daimler AG entwickelt, 
um das kontinuierliche Verhalten eingebetteter Systeme testen zu können. Die ersten Versionen gibt es schon im Jahre 2000.
Es wird jahrelang bei der Daimler AG in der Fahrzeugentwicklung benutzt und weiterentwickelt. Heutzutage besitzt \ac{tpt} die Firma PikeTec. 
Es werden eine Anbindung zu verschiedenen Programmen wie Autosar, Matlab, Labcar und Targetlink und verschiedene Testarten wie \ac{sil}, \ac{hil} \ac{mil},
\ac{pil} unterstützt. Das Programm ist von der Anforderungsanalyse über die Testfallerstellung bis zur Auswertung der Ergebnisse einsetzbar \cite[vgl.][]{piketecwebseite}\cite[vgl.][]{tptwikipedia}.
%Es besitzt einen Support zu verschiedenen Porgrammen wie Autosar, Matlab, Labcar, TargetLink etc
\ac{tpt} ist \cite[vgl.][S. 120]{tpt}:
\begin{itemize}
\item plattformunabhängig: Plattformen können ausgetauscht werden, ohne dass Testfälle neu geschrieben werden müssen
%dadurch können Platformen ausgetauscht werden, ohne dass die Testfälle neu geschrieben werden
\item echtzeitfähig: Echtzeitanforderungen eines eingebetteten Systems werden erfüllt
%\item echtzeitfähig: bei einem Echtzeitbetriebssystem soll die Testdurchführung Echtzeitanforderungen erfüllen
%Echtzeitanforderungen eines eingebetteten Systems werden erfüllt
\item reaktiv: beim Eintritt eines Systemzustands kann der Testfall darauf reagieren
\end{itemize}
Nun soll \ac{tpt} bei \ac{zf} in der Abteilung Softwareentwicklung elektrische Antriebsfunktionen eingeführt werden.
Es soll als einheitliche Testumgebung für \ac{sil} dienen und andere Tools wie Softcar, das ein \ac{zf} internes Tool ist, ablösen.



%Dadurch sollen Einschänkungen von Softcar(siehe Abschnitt XY) gelöst werden.



% Im Rahmen einer Dissertation von Eckard Lehmann wurde TPT bei Daimler entwickelt 
 % Das Ziel von TPT ist es, einen systematischen Ansatz für den Test eingebetteter Systeme zu erarbeiten.
% verbunden sind und dadurch mit 
% von realen Umgebungen 
% beeinflusst werden
 % Eingebettete Systeme beeinflussen reale umgebungen und kommunizieren mit dieser Umgebung in der Regel über 
% kontinuierliche Größen. Das Ziel der Arbeit war, dass ein spezialisierter systematischer Ansatz gegeben ist um eingebettete Systeme zu testen.
% Mit TPT kann man den ganzen Ablauf eines Tests von der Testfallerstellung bis zur Anforderungsanalyse alles abdecken.
% Die ersten Versionen gab es schon 2000. TPT wurde jahrelang von Daimler für ihre eigene Fahrzeugentwicklung benutzt. Heutzutage besitzt piketec 
% TPT und entwickelt es weiter. (Piketec Webseite(https://piketec.com/de/tpt/))
% TPT ist:
% \begin{itemize}
% \item plattformunabhängig

% \item echtzeitfähig

% \item reaktiv
% \end{itemize}

% 20 Jahre nach der ersten Version wird TPT bei ZF eingeführt werden.
% Es soll als einheitliche Testumgebung für \ac{sil} dienen und andere Tools wie Softcar, was ein ZF interenes Tool ist, ablösen.
% Dadurch sollen Einschänkungen von Softcar(siehe Abschnitt XY) gelöst werden.
% \ac{tpt} heißt auf Deutsch übersetzt
% TPT steht für? Englisch: Time Partition Testing  Deutsch:\\

% Was ist TPT?\\
% TPT wurde von Eckard Lehmann in seiner Dissertation bei Daimler entwickelt (von 2004)
% Daimler hat jahrelang selbst TPT weiterentwickelt. Heutzutage besitzt piketec TPT und entwickelt es weiter.
% Piketec Webseite(https://piketec.com/de/tpt/)
% "
% TPT unterstützt alle Testaktivitäten von der Testfallerstellung/-Generierung, der Testausführung, der Testauswertung und dem Testreporting, sowie dem Testmanagement und der Anforderungsanalyse."
% \\
% TPT besitzt auch großen Support zu verschiedenen Porgrammen wie Autosar, Matlab, Labcar, TargetLink etc (
 % "MiL Testing, SiL Testing, PiL Testing, HiL Testing, ECU Testing und Fahrzeugtests ausgeführt werden")
% \\
% Aus <https://piketec.com/de/tpt/> 

% TPT wurde von Daimler für ihre eigene Fahrzeigentwicklung ursprünglich entwickelt.


% Kurz erklären, was kann TPT alles?\\

% Warum wurde TPT entwickelt?\\
% TPT Ziel: Test von eingebetteten Systemen
% Zu dieser Zeit noch keine spezialisierte systematische Ansätze eingebettete Systeme zu testen
% Schwierigkeit embedded systems beeinflussen reale Umgebungen und kommunizieren mit dieser Umgebung in der Regel über kontinuierliche Größen(Sensoren)

% % (
% % Black-box, Zustände werden als Eingänge aufgefasst, sodass bei der Testdurchführung ein bestimmter Zustand eintritt, der gestestet werden soll

% % "Hat ein System beispielsweise drei Be-
% % triebsmodi, in denen es sich jeweils unterschiedlich verhÄalt, so wird fÄur einen
% % Testfall beschrieben, in welchem Betriebsmodus dieser Testfall durchzufÄuhren
% % ist."
% % )

% TPT ist (Kapitel 6)\\
% Plattformunabhängig:\\

% Echtzeitfähig:\\

% Reaktiv:\\


% Warum wird TPT bei ZF eingeführt?\\
% Eine einheitliche Testumgebung für SIL, Ablösen von Softcar und anderen Testumgebungen(Tessy)
% Eine Testumgebung, die alle Tests von Softcar und Tessy durchführen kann und Probleme von Softcar behebt(wie zum Beispiel manuelle Debug Ausgaben in Softcar), 
% für TPT müssen nicht Debug Ausgaben manuell geschrieben werden


% %Quelle Dissertattion TPT\\