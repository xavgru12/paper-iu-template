Antriebsfunktionen elektrischer Maschinen sind als Software geschrieben, laufen auf einer 
Hardware und übernehmen die Steuerung der Maschine. Das macht ein eingebettetes System aus.
Ein eingebettetes System ist in der Regel ein in-the-Loop System. In einem in-the-Loop System bilden Ausgänge des Systems
eine Rückkopplung zu den Eingängen. Dies muss auch beim Testen berücksichtigt werden.
Software-in-the-Loop Test bedeutet, dass die Software als in-the-Loop System getestet wird, wobei die Hardware
simuliert wird. Bei Hardware-in-the-Loop hingegen ist die Hardware vorhanden und die Software wird, während
es auf der Hardware ausgeführt wird, getestet.
Mit SIL wird eine Möglichkeiten geschaffen Software ohne Hardware zu testen. Damit können Kosten von Hardware eingespart werden.
%SIL ist in der Entwicklung bedeutend, da es keine Hardware zum Testen benötigt. Damit können Kosten von Hardware eingespart werden.
Des Weiteren kann dadurch auch in den frühen Phasen der Entwicklung getestet werden \cite[vgl.][S. 1 f.]{silest}\cite[vgl.][]{hilwikipedia}. Es ist wichtig Fehler schnell zu beheben,
denn je länger ein Fehler unentdeckt bleibt, umso teurer wird dieser Fehler \cite[vgl.][]{fehler}.

Diese Arbeit handelt davon TPT als Programm (siehe Abschnitt 1.4) einzusetzen, um einen Schnittstellentest zu überführen (siehe Kapitel 5) und ein
neues Testkonzept mit Äquivalenzklassen (siehe Kapitel 6) zu erarbeiten.




% Beschreibung software-in-the-Loop für elektrische Ant
% in the Loop allgemein -> zur Steuerung elektrischer Maschinen\\
% software-in-the-Loop -> Virtualisierung der Hardware\\
% Schnittstellentest, Äquivalenzklassentest

% Was ist software-in-the-loop? Wichtigkeit von Testen allgemein
% SIL; HIL, MIL
% Was bedeutet in the loop? Warum die Namensgebung?
% Seit wann gibt es SIL, warum wurde es eingeführt? Nutzen?

% Einordnung der Bachelorarbeit, welchen Bereich umfasst die Bachelorarbeit
% Interesse wecken, 
% genaue Beschreibung des Themas
% Ziele der Bachelorarbeit
% Nutzen der Bachelorarbeit für ZF, die Wissenschaft