\section*{Schnittstellentest}
Es ist nicht gelungen die Laufzeit des Quicktests zu verbessern.
Es ist noch ein großes Potenzial vorhanden, denn TPT muss noch besser für das Softwareprojekt konfiguriert werden.
Derzeit ist es nicht möglich einzelne Dateien von TPT analysieren und kompilieren zu lassen. Da die Schnittstellen
jeweils nur in einer Datei einer Komponente definiert sind, könnte sehr viel Laufzeit eingespart werden.
Das Jython Skript, das die Testfälle für den Quicktest automatisiert generiert, funktioniert soweit. Es legt Ordner an,
so wie es im Softwareprojekt auch der Fall ist, sodass sich Entwickler schnell zurecht finden können.
Es ist zwar nicht die Gesamtlaufzeit verbessert worden, aber es ist jetzt möglich, dass einzelne Komponenten oder gar Schnittstellen alleine getestet
werden.% TPT gibt beim Quicktest einen Fehler aus, Windows Fehler. 
% jeweils nur eine Datei 

% Nicht gelungen den Quicktest zu optimieren(Zeiteffizienz verbessern)
% Testausführung dauert sehr lange, Grund; GSIL hat Probleme in TPT Konfiguration

% Automatisierung der Testfälle, hat gut funktioniert mit TPTAPI
% keinen Parser gebaut, der Funktionsaufrufe für Quicktest findet
% Generell alle Funktionen von den Task.c auf Schedule stellen, beziehungsweise Artefakte Ordner benutzen
% Probleme mit Windows Fehler.. ansonsten funktioniert Quicktest 

\section*{Äquivalenzklassentest}
Es wurde eine gute Grundlage geschaffen, sodass ein Äquivalenzklassentest in naher Zukunft eingeführt werden kann.
Es wurde ein Konzept erarbeitet, das man schnell einführen kann. Es wurde gezeigt, wie \ac{sil} Tests überprüft werden kann. 
Es wurde gezeigt wie Testfälle mit Äquivalenzklassen von TPT generiert werden können. Die große Arbeit ist alle
Äquivalenzklassen für jedes Signal zu definieren. Durch das Definieren wird eine Datenbasis geschaffen, die auch für
andere Tests eingesetzt werden kann. 
% Konzept aufgezeigt, aufgezeigt, dass es machbar ist und auch schnell einsetzbar durch Generieren von
% Äquivalenzklassen 
% schnell einsetzbar, beim Überprüfen der SIl Tests müssen nur alle Signale, die im Test verwendet werden
% als Assesslet Equi angegeben werden plus Definition der Äquivalenzklassen
% Es ist denkbar, ein Skript zu schreiben, das alle Signale, die im Testfall vorkommen, als  Äquivalenzklassen
% Assessment auswählt.