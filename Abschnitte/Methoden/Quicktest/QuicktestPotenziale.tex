%\newpage
Ein Potenzial des Quicktests ist, dass die Laufzeit verbessert werden kann. Um die Schnittstellen zu überprüfen, ist es
nicht nötig das Gesamtprojekt zu kompilieren, da jeweils nur eine C Datei einer Komponente benötigt wird. 
Um es möglich zu machen, dass jeweils 
nur eine C Datei pro Komponente analysiert wird, muss die Verbindung zu anderen Dateien getrennt und mit einem Platzhalter ersetzt werden.
Dieses Abtrennen wird auch Stubben genannt \parencite[S. 337]{integration}.
Nach dem Status Quo ist das Stubben nicht möglich, da das Gesamtprojekt zu 
einer \ac{exe} kompiliert wird. Beim Kompilieren zu einer \ac{exe} werden alle Abhängigkeiten benötigt und es kann nicht gestubbt werden.
Mit TPT können Dateien gestubbt werden.

% Das bestehende Konzept hinter dem Quicktest ist, dass das Gesamtprojekt inklusive der Quicktest.c kompiliert und gelinkt wird. 
% Es werden alle Sourcen kompiliert, da der Compiler alle Abhängigkeiten benötigt. Aufgrund der Größe des Projekts dauert das Kompilieren
% länger als eine Stunde.
% %(Die daraus resultierende exe wird ausgeführt und ein log Datei wird beschrieben.)
% Das große Potenzial besteht darin, dass es
% nicht nötig ist, das Gesamtprojekt zu kompilieren, da pro Unit jeweils nur eine einzige Source benötigt wird. Es wird eine 
% Möglichkeit gesucht, dass nur die bestimmten Sourcen auf die Schnittstellen überprüft werden. TPT bietet dafür die Option,
% dass für den Schnittstellentest unbedeutende Dateien gestubbt werden. 

Ein weiteres Potenzial ist, dass ein Ordner mit Artefakten im Projekt hinfällig gemacht werden kann und somit 
auch nicht gepflegt werden muss. Es gibt nämlich einen Ordner, in denen Funktionsaufrufe des Quicktests je Komponente
in Text-Dateien stehen. Indem ein Parser sich für den Quicktest automatisiert die Funktionsaufrufe zusammensucht, 
müssen die Dateien nicht mehr abgelegt werden.
Dadurch erreicht man, dass die Funktionsaufrufe immer up to date sind und nicht mehr händisch gepflegt werden müssen.
% Die Funktionsaufrufe für den Quicktest liegen in
% einem Ordner dauerhaft ab. Vermutlich wurde die Liste einmal manuell erstellt. Dies ist nicht von Vorteil, da bei 
% dementsprechenden Veränderungen auch die Liste mit den Funktionsaufrufen per Hand gepflegt werden müssen. Es wäre
% sinnvoll diese Informationen automatisiert zu bekommen. Damit wären die Funktionsaufrufe immer up to date und es
% wären die Artefakte auch nicht im Projekt abgelegt.








% Gründe, warum Quicktest langsam ist:\\
% Alle Sourcen werden kompiliert\\

% Es wird eine exe erzeugt, die Softcar übergeben wird\\
% Um eine exe zu erhalten müssen ALLE Sourcen kompiliert werden, da Compiler alle Abhängigkeiten(includes etc) benötigt\\


% Einmaliger Funktionsaufruf, um Werte einer FU Task zu übergeben, anstatt für jedes Label einzeln den Funktionsaufruf zu starten\\

% Eine exe wird erzeugt, da Softcar eine exe benötigt. TPT ebnötigt keine exe. TPT kann Fehlende Softwareteile stubben.\\


% Artefakte (Funktionsaufrufe) werden in einem Ordner abgespeichert und dann vom C Sharp Skript benutzt. 
% Das macht man nicht! Es sollten die Infos für die Funktion geparst werden und dann im Skript benutzt. Diese Infos müssen nicht in einem Ordner dauerhaft gespeichert werden