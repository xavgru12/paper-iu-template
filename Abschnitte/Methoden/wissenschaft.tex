Quelle über dynamischer Test,black box (Äquivalenzklassen): Basiswissen\_Softwaretest...pdf\\

% Von Y:\Bachelor\Bachelorarbeit\Basiswissen_Softwaretest_Aus-_und_Weiterbildung_zu..._----_(Pg_131--193).pdf

% Bis Seite 148 geht es um Blackbox-Verfahren

% Ziel des Testens: Nachweis der Erfüllung der festgelegten Anforderungen

Dynamischer Test -> Blackbox -> Äquivalenzklassen\\

Systematische Erstellung von Testfällen:
% Blackbox Verfahren und whitebox Verfahren
% "Die vorausgesagten Ergebnisse (Ausgaben, Änderung von internen
% Zuständen usw.) müssen vor der Ausführung der Testfälle feststehen
% und dokumentiert werden."
% (In Softcar, der setpoint wird berechnet durch die Eingangswerte
% Danach wird überprüft, ob das Programm auf das gleiche Ergebnis wie der setpoint gekommen ist)

"Bei den Blackbox-Verfahren wird das Testobjekt als schwarzer
Kasten angesehen. Über den Programmtext und den inneren Aufbau
sind keine Informationen notwendig."\\

"Bei den Whitebox-Verfahren wird auch auf den Programmtext
zurückgegriffen. Während der Ausführung der Testfälle wird der
innere Ablauf im Testobjekt analysiert (der Point of Observation liegt
innerhalb des Testobjekts)."\\

Blackbox: spezifikationsbasiertes Testentwurfsverfahren\\

Whitebox: strukturelles Testverfahren\\

dynamischer vs statischer Test\\

Diagramm zeichnen mit der Einordnung von den unterschiedlichen Testarten(dynamisch, statisch, black box etc.)


% Trends in Software Testing

% SILreliability
% SILreliability Vergleich von sil und hil, Trend geht dazu, dass mehr Hardware Sachen mit Software gelöst werden, V Modell zeigt sil ist für module und auch Integrationstest, Vorteil von sil gegenüber Hil ist, dass mit sil kein corrupted data entstehen kann (unter 4. Results)

% \begin{lstlisting}
% Virtual platforms von Software-in-the-Loop_simulation_of_embedded_control_applications_based_on_Virtual_Platforms.pdf
% \end{lstlisting}
% Ist ein Trend zur embedded software Entwicklung

% "Another advantage
% of using virtual platforms in the early stage of the software
% development is that they allow observing the behavior of the
% entire system which leads to excellent debugging options
% compared to the “black box”-like behavior of embedded
% devices."


% Another advantage
% of using virtual platforms in the early stage of the software
% development is that they allow observing the behavior of the
% entire system which leads to excellent debugging options
% compared to the “black box”-like behavior of embedded
% devices.