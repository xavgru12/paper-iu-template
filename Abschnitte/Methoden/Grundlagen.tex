In diesem Kapitel werden die Grundlagen erarbeitet, worauf die Umsetzung basieren wird.
Es wird auf Schnittstellen eingegangen und wie diese gestestet werden können. Ein weiterer Punkt sind
die Äquivalenzklassen. Dabei werden die Äquivalenzklassen mathematisch beschrieben und erklärt, wie sie in 
der Informatik angewandt werden.
 % Zuerst werden diese mathematisch beschrieben und danach wie sie in der Informatik
% eingesetzt werden. 
% Für den Schnittstellentest ein Punkt, für die Äquivalenzklassen.
Die Umsetzung erfolgt in \ac{tpt}, weshalb im Abschnitt 2.3 erklärt wird, welche Funktionalitäten \ac{tpt} hat und wie es 
konfiguriert wird.