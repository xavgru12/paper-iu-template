\section*{Zusammenfassung}
Der bestehende Quicktest von ZF wird analysiert und in TPT übertragen. Es wird ein Code nachgestellt, in der die
Schnittstelle definiert ist.
Es wird ein Skript in Jython
geschrieben, das automatisiert die Testfälle in TPT generiert.

Es wird ein Konzept für den Äquivalenzklassentest vorgestellt. Der Äquivalenzklassentest ist eine gute Erweiterung,
insbesondere für bestehende \ac{sil} Tests. Es wird auch gezeigt, wie Äquivalenzklassen selbst überprüft werden können.
Es wird erklärt, wie Äquivalenzklassen gebildet werden können.
% Durch das Definieren der Äquivalenzklassen wird eine Datenbasis geschaffen, die für weitere Tests in TPT
% eingesetzt werden können. 
% Schnittstellentest, alles gemacht

% Äquivalenzklassentest alles gemacht
% schnell einsetzbar

\section*{Ausblick}
Als Potenzial für den Quicktest wird in Kapitel 3 genannt, dass ein Parser Funktionsaufrufe, in denen die Schnittstellen
definiert sind, finden könnte. Dadurch müssen die Funtionsaufrufe nicht mehr in einem Ordner abgespeichert 
werden. Dieser Parser wurde noch nicht umgesetzt. 
TPT muss weiter für die bestehende Software konfiguriert werden. Wenn es durch die Konfiguration möglich ist, 
dass nur die Dateien, die für den Quicktest nötig sind, kompiliert werden, so ergibt sich auch eine schnellere 
Laufzeit des Quicktests in TPT. Die Äquivalenzklassen müssen für jedes Signal erstellt werden, um Äquivalenzklassentests
einführen zu können.
 % Es ist
% Um den Äquivalenzklassentest einzusetzen, um bestehende SIL Tests zu überprüfen, so 
% Parser der Funktionsaufrufe bauen für Quicktest
% Quicktest Fehler beheben.
% Äquivalenzklassen definieren
% Äquivalenzklassen definieren plus in SIL TPT Assesslets richtig einsetzen(alle Signale in Assesslet von SIL Tests
% angeben)