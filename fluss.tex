
% Mathematikumgebungen von der AMS laden
%\usepackage{amsmath}
%\usepackage{amssymb}

% TikZ laden
%\usepackage{tikz}

% Verwendete TikZ-Bibliotheken laden
%\usetikzlibrary{positioning,shapes.geometric}


%\textwidth 16cm
%\hoffset -2.7cm
%\pagestyle{empty}
Um mein Skript, insbesondere das Auflösen der Abhängigkeiten der \textit{FUs} zu veranschaulichen, ist im Folgenden ein
Flussdiagramm angehängt.

  \begin{figure}[!h]
    \centering


  \begin{tikzpicture}%[every node/.style={rectangle,draw,text width=14em,align=center}]
    [
      % Stil für Ein- und Ausgabe
      io/.style={trapezium, trapezium left angle=70, trapezium right angle=110, fill=magenta!10, draw=magenta},
      % Stil für Operationen
      op/.style={rectangle, fill=orange!10, draw=orange},
      op4/.style={rectangle, fill=green!20, draw=green},
			op2/.style={rectangle, fill=green!20, draw=green, minimum height=2cm, minimum width=38mm},
			op3/.style={rectangle, fill=orange!10, draw=orange, minimum height=2cm, minimum width=38mm},
      % Stil für Entscheidungen
      cn/.style={diamond, aspect=2, inner sep=1pt, fill=red!10, draw=red, minimum width=20mm, minimum height=14mm},
      cn2/.style={diamond, aspect=2, inner sep=1pt, fill=blue!20, draw=blue, minimum width=20mm, minimum height=14mm},% Distanz zwischen den Knoten
      node distance=4mm]
    % Knoten
    \node[op4] (class) {create class};
    
    \node[op4, below=of class] (init) {initialize object};
    \node[op4, below=of init] (list) {put object in new list and select first item};
    %\node[op, below=of list] (firstitem) {select first item in list};
    \node[cn2, below=of list] (cond1) {};
    \node[op4, below=of cond1] (finddep) {object.findDependencies()};
    \node[op4, below=of finddep] (depnewlist) {put dependencies in new list and select first item};
    \node[cn2, below=of depnewlist] (cond2) {};
    %\node[op, right=of cond2] (wrongdep) {select next item};
    %\node[op, below=of cond2] (makefiles) {search for item in dictionary to get makefiles};
    \node[op4, below=of cond2] (initdep) {initialize object using item};
    \node[op4, below=of initdep] (addobject) {add to object list};
    \node[cn2, below=of addobject] (cond3) {\shortstack{last item from \\ dependency list?}};
    \node[op4, right=of cond3] (wrongdep) {select next item};
    \node[op4, below=of cond3] (findsubs) {object.findSubdirectories()};
    \node[op4, below=of findsubs] (findfiles) {object.findFiles()};
    \node[cn2, below=of findfiles] (cond4) {\shortstack{last item from\\ object list?}};
    \node[op4, left=of cond4] (wrongobject) {select next item};
    \node[op4, below=of cond4] (end) {Ende};
    
    \path[->]
      (class) edge (init)
			(init) edge (list)
			(list) edge (cond1)
      (cond1) edge (finddep)
      (finddep) edge (depnewlist)
      (depnewlist) edge (cond2)
      (cond2) edge (initdep)
      (initdep) edge (addobject)
      (addobject) edge (cond3)
      (cond3) edge node[above=0.2cm] {No}(wrongdep)
      (cond3) edge node[right=0.2cm] {Yes} (findsubs)
      (findsubs) edge (findfiles)
			(findfiles) edge (cond4)
      (cond4) edge node[above=0.2cm] {No}(wrongobject)
      (cond4) edge node[left=0.5cm] {Yes} (end);

    \draw[->] 
      (wrongdep) --  ++(2,0) |- (cond2)
      (wrongobject) -- ++(-2,0) |- (cond1);
		% \draw[->] (cond2) -- node[below] {Nein} ++(2.9,0) -- (entscheid);
		% \draw[->] (cond4) -- node[below] {Nein} ++(4.0,0) -| (gescheitert2);
		% \draw[->] (cond3) -- node[below] {Nein} ++(-4.0,0) -| (gescheitert3);
      %(wrongdep) edge (cond1);
			% (cond1) edge node[right] {Ja} (cond2)
			% (cond2) edge node[below] {Ja} (gesetz2);
    
    % \node[cn, align=center, below=of begehren] (cond1) {okay\\Unterschriften?};
		% \node[cn, align=center, below=of cond1] (cond2) {Landtag nimmt\\Gesetzentwurf an?};
		% \node[op3, align=center, right=of cond1] (gescheitert) {Volksbegehren\\gescheitert};
		% \node[op, align=center, right=of cond2] (entscheid) {Volksentscheid};
		% \node[io, below=of entscheid] (abstimmung) {Stimmen Sie ab!};
		% \node[cn, align=center, below=of abstimmung] (cond3) {Mehrheit stimmt\\mit Ja?};
		% \node[cn, align=center, below=of cond3] (cond4) {Mehr als ein Drittel\\aller Stimmberechtigten\\ stimmt mit Ja?};
		% \node[op2, align=center, below=of cond4] (gesetz) {Gesetz};
		% \node[op3, align=center, right=of gesetz] (gescheitert2) {Volksbegehren\\gescheitert};
		% \node[op3, align=center, left=of gesetz] (gescheitert3) {Volksbegehren\\gescheitert};
		% \node[op2, align=center, left=of cond2] (gesetz2) {Gesetz};
    % % Kanten
    % \path[->]
    %   (begehren) edge (cond1)
		% 	(entscheid) edge (abstimmung)
		% 	(abstimmung) edge (cond3)
    %   (cond3) edge node[right] {Ja} (cond4)
		% 	(cond4) edge node[right] {Ja} (gesetz)
		% 	(cond1) edge node[right] {Ja} (cond2)
		% 	(cond2) edge node[below] {Ja} (gesetz2);

    % \draw[->] (cond1) -- node[below] {Nein} ++(2.8,0) -- (gescheitert);
		% \draw[->] (cond2) -- node[below] {Nein} ++(2.9,0) -- (entscheid);
		% \draw[->] (cond4) -- node[below] {Nein} ++(4.0,0) -| (gescheitert2);
		% \draw[->] (cond3) -- node[below] {Nein} ++(-4.0,0) -| (gescheitert3);

  \end{tikzpicture}
  \caption{Flussdiagramm Auflösen der Abhängigkeiten}
  \end{figure}

